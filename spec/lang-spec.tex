\documentclass[10pt]{article}

% Lines beginning with the percent sign are comments
% This file has been commented to help you understand more about LaTeX

% DO NOT EDIT THE LINES BETWEEN THE TWO LONG HORIZONTAL LINES

%---------------------------------------------------------------------------------------------------------

% Packages add extra functionality.
\usepackage{
	times,
	graphicx,
	epstopdf,
	fancyhdr,
	amsfonts,
	amsthm,
	amsmath,
	algorithm,
	algorithmic,
	xspace,
	hyperref,
	comment}
\usepackage[left=1in,top=1in,right=1in,bottom=1in]{geometry}
\usepackage{sect sty}	%For centering section headings
\usepackage{enumerate}	%Allows more labeling options for enumerate environments 
\usepackage{epsfig}
\usepackage[space]{grffile}
\usepackage{booktabs}
\usepackage{amsmath}
\usepackage[super]{nth}
\usepackage{array}

% This will set LaTeX to look for figures in the same directory as the .tex file
\graphicspath{.} % The dot means current directory.

\pagestyle{fancy}

\lhead{\YOURID}
\chead{\MyLang: Language Specification}
\rhead{\today}
\lfoot{CSCI 334: Principles of Programming Languages}
\cfoot{\thepage}
\rfoot{Spring 2022}

% Some commands for changing header and footer format
\renewcommand{\headrulewidth}{0.4pt}
\renewcommand{\headwidth}{\textwidth}
\renewcommand{\footrulewidth}{0.4pt}

% These let you use common environments
\newtheorem{claim}{Claim}
\newtheorem{definition}{Definition}
\newtheorem{theorem}{Theorem}
\newtheorem{lemma}{Lemma}
\newtheorem{observation}{Observation}
\newtheorem{question}{Question}

\setlength{\parindent}{0cm}
%---------------------------------------------------------------------------------------------------------

% DON'T CHANGE ANYTHING ABOVE HERE
\setlength{\parindent}{1cm}
% Edit below as instructed
\newcommand{\MyLang}{G$\flat$}
\newcommand{\PartnerOne}{Ezra Joffe-Hancock}
\newcommand{\PartnerTwo}{Zach Stein-Perlman}
\newcommand{\YOURID}{\PartnerOne{} + \PartnerTwo{}}

\title{\MyLang: Language Specification}
\date{Spring 2022}
\author{\PartnerOne{} and \PartnerTwo{}}

\begin{document}
\maketitle

\vspace{\baselineskip}

\section{Introduction}
Ezra plays alto sax in a jazz combo here, and when he's struggling with soloing over certain songs, it helps him to run over the chord changes and play along with the recording, but the best is when there's a backing track on YouTube, so he can play along with the chords without drowning whoever is soloing on the recording. However, there are often not backing tracks for songs, or only backing tracks that are played in certain keys, so there's a need for more flexibility and availability of backing tracks. There are music notation software systems that allow you to write songs and chords, but these often take a while to write, as they're designed for a greater range of features. In other words, they're too flexible and open-ended for the purposes of a quick backing track.

We created a language that makes it easy to quickly create a backing track to play over. The programmer creates sections of music by arranging chords and choosing their durations, then describes how many times and in what order to play each section. The output is an XML file that describes the corresponding audio. MusicXML is a standard sheet music file representation that can be interpreted by many different third-party programs, and contains information that allows the music to be both visually displayed and produce audio output (as it is MIDI compatible). We leave the decision of what program to use to interpret the XML file up to the user, as there is a large range available, but note that SoundSlice (\url{https://www.soundslice.com/}) is an online free program with a neat user interface that does so (go to the website, press new slice, then "upload notation").


\section{Design Principles}
The language is simple and accessible for a non-technical user, and it should be easy to quickly translate an existing written chord progression into a program. The syntax is natural. The order of commands makes programs easily readable with intuitive layout. Simplicity in implementation is prioritized (worse is better!). A valid program is ideally easy to read and facilitates quickly understanding the music that it describes.

\section{Example Programs}

Examples can be run by calling
\begin{verbatim} dotnet run <input file> \end{verbatim}
on the command line from inside the Gflat directory inside the lang directory. An XML file matching the name of the example is created in the Gflat directory, which can be put into various programs (including soundSlice) to produce musical notation or sound output. Below we see example Gflat programs highlighting different language features.

Example 1: simplest possible
\begin{verbatim}
let Song = {
Cmaj 1
}
play Song 1
\end{verbatim}

Example 2: variety of chords and repeated chorus
\begin{verbatim}
let Chorus = {
Emaj 2
G#min 2
Bdim 2
Eaug 4
Db7 2
Cmaj7 2
F#min7 2
}
play Chorus 2
\end{verbatim}

Example 3: multiple sections defined, sections reused in play
\begin{verbatim}
let Intro = {
Cmaj 2
}

let Chorus = {
Cmaj 1
G7 1
}

play Intro 1 Chorus 10 Intro 1
\end{verbatim}

Example 4: sections used within other sections, number of times they repeat specified
\begin{verbatim}
let Intro = {
Cmaj 2
}

let Phrase = {
D7 1
A7 2
Fmin7 3
}

let Chorus = {
Phrase 1 
Cmaj 1
G7 1
Phrase 2
}

play Intro 1 Chorus 10 Intro 1
\end{verbatim}
\section{Language Concepts}
In order to program in the language, a user should have a basic understanding of chords (so they can specify them by writing out their letters, and understanding different chord qualities) and an understanding of a beats as a way to think about the duration of chords. They should understand also how to transfer information from a written piece of sheet music with chords to information within the language, by copying down chords and putting them in different sections as appropriate to efficiently capture the structure of a song.


\section{Syntax}
Programs have two parts: defining sections and playing them. When defining a section, we give the name of the section, and its components: chord-duration pairs and references to other sections and how many times to repeat them. A chord is made up of a note (representing the starting note of the chord) and a chord quality (either Major, Minor, Diminished, Augmented, Dominant 7th, Major 7th, or Minor 7th). A chord is followed by an integer specifying the number of beats we want the chord to be played for. In the play section of the program, we refer to an order of section names and the number of times we want each one to be played. If a section is defined multiple times within a program (i.e. a section with the same name is assigned more than once) only the latest definition will be used (thus the practice should be avoided). 



\begin{verbatim}
<expr>       ::= <assignment>+ <play>
<assignment> ::= let ␣ <variable> <ws> = <section>
<section>    ::= <ws> { noise+ } <ws>
<noise>      ::= <ws> <subsection> <ws> | <ws> <sound> <ws>
<subsection> ::= <ws> <variable> <ws> <int> <ws>
<sound>      ::= <chord> <int>
<chord>      ::= <note> <ws> <quality> <ws>
<ws>         ::= ␣*
<note>       ::= A | A# | Bb | B | C | C# | Db | D | D# | Eb | E | F | F# 
               | Gb | G | G# | Ab
<quality>    ::= maj | min | dim | aug | 7 | maj7 | min7
<variable>   ::= <string>
<play>       ::= play content+
<content>    ::= <ws> <variable> <ws> <int> <ws>
\end{verbatim}
Note that \begin{verbatim}<variable>\end{verbatim} may not be ``play" or ``let". Encoding this in BNF would be long and unenlightening, so we just mention it here.
% If we did not allow redefining sections---if defining Chorus and then defining Chorus was invalid---we would say:
% The grammar of G$\flat$ appears not to be context-free---the language is too powerful, letting users define section names!---so we must mention outside the above description first that each \begin{verbatim}<variable>\end{verbatim} must be distinct (and not be ``play" or ``let") and second that each \begin{verbatim}<variable>\end{verbatim} in \begin{verbatim}<play>\end{verbatim} must be defined in \begin{verbatim}<assignments>\end{verbatim}

\section{Semantics}

Primitives in G$\flat$ include notes, which are every type of chromatic note with equivalent notes represented both as flats and sharps (e.g. G\# and A$\flat$) but not when there is no proper flat or sharp version of the note (e.g. E\# and F$\flat$). Chord qualities are a primitive. Section names are another primitive, and are of string. Another primitive are integers, which are used specify the duration of chords, the number of times a section repeats when referred to in another section, and the number of times a section repeats when referred to in play.



\begin{center}
\begin{tabular}{c|c|c|c|p{4cm}} 
Syntax & Abstract syntax & Type & Prec./Assoc. & Meaning \\ [0.5ex] 
\hline
C & Note & Note & n/a & A note \\ \hline
Cmaj & Quality of Note & Chord & n/a & A chord to be played where C stands for the note and maj for the quality \\ \hline
Cmaj 1 & Sound of Chord * int & Expr & n/a & A chord and its duration \\ \hline
Chorus & Variable of string & Expr & n/a & The name of a section \\ \hline
Chorus 1 & Subsection of Expr * int & Expr & n/a & A section and how many times to repeat it. \\ \hline
sound1 ... soundn & Section of Expr list & Expr & n/a & A sequence of sounds and subsections \\ \hline
let var = \{section\} & Assignment of Expr * Expr & Expr & n/a & A pairing of a section with a name \\ \hline
% assign1 ... assignn & Assignments of Assignment list & Assignment list → Assignments & n/a & A list of sections and their associated names \\ \hline
play var1 x1 ... varn xn & Play of (Expr * int) list & Expr & n/a & A specification of sections to play \\ \hline
$<$Assignment list$>$ $<$Play$>$ & Program of Expr list * Expr & Expr & n/a & A complete program \\ 
\end{tabular}
\end{center}

\section{Remaining Work}

Potential extensions:

- Implement a program correctness checker; e.g., allowing for inconsistencies in capitalization

- Add instruments beyond piano

- Allow more flexible control over tempo, volume, and octave

- Allow for individual notes and custom chords (where the user specifies all the notes in them if they are not a standard chord)
% We could easily implement a simple "tonic" chord quality; with more work we could allow arbitrary user-defined chords

- Chord names printed above chords in musical notation representation
% - Pretty printing of musical notation corresponding to the audio
% ^We can actually get this from the XML too.

% DO NOT DELETE ANYTHING BELOW THIS LINE
\end{document}